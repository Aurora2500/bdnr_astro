\documentclass[spanish]{article}
\usepackage{csquotes}
\usepackage[spanish]{babel}
\selectlanguage{spanish}
\usepackage[utf8]{inputenc}
\usepackage{authblk}
\usepackage{amsmath}
\usepackage{amsfonts}

\usepackage[
	backend=biber,
	style=numeric,
]{biblatex}

\usepackage{enumitem}
\usepackage{extarrows}
\usepackage{mathtools}
\usepackage{systeme}
\usepackage{graphicx}
\usepackage{float}
\usepackage{listings}
\usepackage{listingsutf8}

\usepackage{multirow}
\usepackage{minted}

\graphicspath{ {./img/} }

%\addbibresource{./sources.bib}

\newcommand{\cimg}[2]{
\begin{figure}[H]
	\center
		\includegraphics[width=#2\linewidth]{#1}
\end{figure}
}

\begin{document}

\begin{titlepage}
	\centering
	{\huge\bfseries Astro \par}
	\vspace{2cm}
	{\scshape\Large Javier Franco González \par\tt{javier.franco101@alu.ulpgc.es}\par}
	\vspace{1cm}
	{\scshape\Large Pablo Guilló Jiménez \par\tt{pablo.guillo101@alu.ulpgc.es}\par}
	\vspace{1cm}
	{\scshape\Large Aurora Zuoris \par\tt{aurora.zuoris101@alu.ulpgc.es}\par}
	\vspace{3cm}
	{\scshape\large Bases de Datos No Relacionales \par}
	\vspace{1cm}
	{\scshape\large Grado en Ciencias e Ingeniería de Datos\par}
	\vspace{1cm}
	{\scshape\large Escuela de Ingeniería Informática\par}
	\vspace{1cm}
	{\scshape\large Universidad de Las Palmas de Gran Canaria\par}
	\vspace{1cm}
	{\scshape\large \today{} \par}
\end{titlepage}

\tableofcontents

\section{Introcucción}

El proyecto Astro consiste en un programa capable de interactuar dinamicamente con la API de space traders,
un juego de simulación de comercio especial. En este sentido, se intenta simular un sistema de comercio
ejemplar que podría existir en un mundo real.
Dado el enfoque de la asignatura, se hace un uso intensivo de bases de datos no relacionales para
cumplir con los objectivos del proyecto.

\section{Descripción del sistema}

\section{Estado del arte}

\section{Analysis y selección de herramientas}

Para el desarollo del código se ha optado por TypeScript ejecutado en Node.js.
TypeScript es un lenguaje de programación basado en JavaScript que añade tipos estáticos opcionales a este.
Esta opción se ha tomado dado que Node.js es un entorno de ejecución de JavaScript basado en un
bucle de eventos asíncronos, lo que lo hace ideal para situaciones con un alto número de operaciones de entrada/salida, como es el caso aquí
con la interacción con la API y con las bases de datos.
Además, el lenguaje es muy popular dandole un gran ecosistema de librerias ya desarolladas para facilitar el uso de las bases de datos utilizadas,
y además es muy fácil desarollar y iterar rápidamente con el progreso del proyecto.

En cuanto a las bases de datos, se ha optado por utilizar Redis, MongoDB y Cassandra.
Redis se ha utilizado para almacenar datos necesarios para el funcionamiento del programa, como por ejemplo los tokens de autenticación.
También se usa para un acceso rápido a datos que se accedan frecuentemente.

MongoDB se utiliza para representar los datos necesarios para las operaciones que se llevan a cabo, por ejemplo el estado de las naves,
los diferentes mercados y los precios de los productos en cada uno de ellos.
MongoDB se usa por su facilidad de uso y su flexibilidad, ya que los datos que se almacenan pueden cambiar con el tiempo.
Además, almacenando todos los datos necesarios para las operaciones significa que el programa
se puede parar y reiniciar sin perder el estado de las operaciones.

Por último, Cassandra se utiliza para almacenar datos históricos de las acciones realizadas y
de precios de productos en los diferentes mercados a lo largo del tiempo.
Se utiliza como una base de datos de solo escritura, ya que los datos no se modifican una vez escritos
ya que representan las acciones realizadas en el pasado, con lo que se conservan para futuros análisis.
Se ha optado por Cassandra para este propósito ya que es una base de datos distribuida y escalable,
permitiendo almacenar grandes cantidades de datos y realizar consultas rápidas sobre ellos,
algo que puede ser util con el paso del tiempo ya que esta seguirá creciendo.

\section{Diseño del sistema}

\section{Incorporación de los datos a la BBDDNR}

\section{Consultas realizadas para cumplir objetivos}

\section{Productos finales obtenidos}

\section{Trabajo futuro}

\section{Conclusiones}

\end{document}